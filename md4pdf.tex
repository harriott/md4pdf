% http://momentary.eu/

% ---------------------------------------------------------------------------------------
% Converting markdown files with the Memoir (article) class allows better heading styles.
% ---------------------------------------------------------------------------------------

% Pandoc defaults to \paragraph{4th level heading}, which causes the 4th level heading to merge into the following text - awkward if it's a poem.  The cure offered here, for markdown files with headings up to 4-levels deep, is to call pandoc like this:
% pandoc -Vdocumentclass:memoir -Vclassoption:article -H <path_to_this_file> -Vmainfont:Arial --toc --toc-depth=4 -f markdown_strict content_file.md -o content_file.pdf --latex-engine=xelatex
% For better page layout, you'll need a modified ~/.pandoc/templates/default.latex
% such as mine  https://github.com/harriott/pandoc-templates

% Using memoir(article) moves headings up one level so that the first is Chapter, which has some snags that this file resolves.  I also add in text before the page number.

% Joseph Harriott, 10/14
% ----------------------

% Remove the Contents announcement (http://texblog.org/2011/09/09/10-ways-to-customize-tocloflot/):
\renewcommand\contentsname{} % could feed in the content_file name in these braces

% Suppress the chapter numbering in the toc (and everywhere, http://www.latex-community.org/forum/viewtopic.php?f=5&t=478):
\renewcommand{\thechapter}{}

% Get rid of Chapter-numbers in the body (somewhat redundantly now) and (more importantly) enlargen the Chapter-header text (http://www.latextemplates.com/forum/discussion/488/how-can-i-get-rid-of-the-first-chapter-header-which-appears-above-the-chapter-title/p1):
\usepackage{titlesec}
\titleformat{\chapter}{}{}{0em}{\bfseries\LARGE}

% Tidy the Chapter-header spacing (http://tex.stackexchange.com/questions/47953/how-can-i-hide-the-chapter-headings-number-on-latex-and-change-the-spacing):
\titlespacing{\chapter}{0pt}{30pt}{*2}

\usepackage{xcolor}
% Use Memoir's inbuilt commands to reformat the footer (could replace "Page" with the file's name):
\makeevenfoot{plain}{}{\textcolor{lightgray}{Page} \thepage\ }{}
 \makeoddfoot{plain}{}{\textcolor{lightgray}{Page} \thepage\ }{}
